    \documentclass{article}
\usepackage[utf8]{inputenc}
\usepackage{amsmath}

\title{Sorular}
\author{VK}

\begin{document}


\maketitle

Bazen çevreden hoşuma giden sorular görüyorum. Bunları da çözüp atmak yerine bir yerde depolayıp örnek sorular oluşturma fikri aklıma geldi. O yüzden Github'ta tutayım dedim, belki gün gelir yeterli soru birikirse millet de faydalanır (faydalanamadı).

\section{Soru: 2020/03/12}
Soru: $\int_0^\infty te^{-kt^2}dt = \frac{1}{2k}$ ise n'in tek degerleri icin $\int_0^\infty t^ne^{-kt^2}dt$ nedir?
\\ \\
\textbf{Cevap: }\\

$n$ tek ise $n = 2m+1 \quad \forall m \in \{0,1,2,\cdots \}$ yazılabilir. Öyleyse integral: 

\begin{equation}
    \int_0^\infty t^{2m+1}e^{-kt^2}dt = \int_0^\infty t^{2m}te^{-kt^2}dt
\end{equation}

Bunu integral reduction ile çözmeliyiz. Ama önce değişken dönüşümü yapalım. $z = t^2$ için $dz = 2tdt$ olur.
\begin{align}
    z &= t^2 \\
    dz&= 2tdt \\
    t^{2m} &= (t^2)^m = z^m \\
    t &\rightarrow 0, \quad z \rightarrow 0 \\
    t &\rightarrow \infty, \quad  z \rightarrow \infty
\end{align}

Öyleyse:

\begin{equation}
     \underbrace{\frac{1}{2} \int_0^\infty z^m e^{-kz}dz}_{I(m)}
\end{equation}

Bundan sonra $I(m)$'i çözmek integral reduction ile çok kolay! Tabi yardımımıza kısmi integrasyon yetişsin. $\int vdu = uv - \int udv$
\begin{align}
    v &= z^m \\
    dv&= mz^{m-1} \\
    du&= e^{-kz}dz \\
    u &= -e^{-kz}/k \\
\end{align}

Yerlerine yazarsak:

\begin{align}
    \label{eq:Im}
    I(m) = \frac{1}{2}\left(\underbrace{-e^{-kz}z^m/k}_{f(z)}\right|_0^\infty + \frac{m}{k}\underbrace{\frac{1}{2}\int_0^\infty z^{m-1}e^{-kz}dz}_{I(m-1)} 
\end{align}

Burada $f(z)$ hesaplanırsa 
\begin{align}
\lim_{z \to \infty} f(z) &= \lim_{z \to \infty} \left(-\frac{z^m}{e^{kz}}\right) = 0 \\
f(0) &= 0 
\end{align}


Böylelikle Denklem \eqref{eq:Im}: 
\begin{align}
    I(m) = \frac{m}{k} I(m-1) 
\end{align}

$m=1$ için soruda verildiği üzere $I(0) = 1/(2k)$ olduğundan: 

\begin{align}
    I(1) &= \frac{1}{k} \cdot \frac{1}{2k} = \frac{1}{2k^2} \\
    I(2) &= \frac{2}{k} \cdot \frac{1}{2k^2} = \frac{1}{k^3} \\
    I(3) &= \frac{3}{k} \cdot \frac{1}{k^3} = \frac{3}{k^4} \\
      & \;\;\vdots \\
    I(m) &= \frac{m!}{2k^{m+1}} 
\end{align}

Burada tabi soruda $n$ verildigi icin $m$ cinsinden birakmak yakışık almaz. Onu da çevirmek lazım (çevirmedi)




\end{document}
